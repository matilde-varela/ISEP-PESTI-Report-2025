% we include the glossary here (frontmatter is included with \input, so this command is as if it was in main.tex)
\newacronym{ESG}{ESG}{Environmental, Social, and Governance }
\newacronym{GRI}{GRI}{Global Reporting Initiative}
\newacronym{SASB}{SASB}{Sustainability Accounting Standards Board}
\newacronym{ONU}{ONU}{Organização das Nações Unidas}
\newacronym{PESTI}{PESTI}{Projecto / Estágio}
\newacronym{PRI}{PRI}{Princípios para Investimento Responsável}
\newacronym{CRS}{CRS}{Corporate Social Responsibility}
\newacronym{PNL}{PNL}{Processamento de Linguagem Natural}
\newacronym{IA}{IA}{Inteligência Artificial}
\newacronym{ODS}{ODS}{Objetivos de Desenvolvimento Sustentável}
\newacronym{UNEP FI}{UNEP FI}{UNEP Finance Initiative}

\frontmatter % Use roman page numbering style (i, ii, iii, iv...) for the pre-content pages

\pagestyle{plain} % Default to the plain heading style until the thesis style is called for the body content

%----------------------------------------------------------------------------------------
%	TITLE PAGE
%----------------------------------------------------------------------------------------

\maketitlepage


%----------------------------------------------------------------------------------------
%	STATEMENT of INTEGRITY
%----------------------------------------------------------------------------------------
\integritystatement

%----------------------------------------------------------------------------------------
%	DEDICATION  (optional)
%----------------------------------------------------------------------------------------
%
\begin{dedicatory}
Dedico este projeto a todos que me permitiram chegar a este capítulo da minha vida académica.
\end{dedicatory}

%----------------------------------------------------------------------------------------
%	ACKNOWLEDGEMENTS (optional)
%----------------------------------------------------------------------------------------

\begin{acknowledgements}

Gostaria de iniciar esta secção do relatório a expressar a minha profunda gratidão às pessoas que me acompanharam ao longo deste percurso e contribuíram para a conceção deste projeto.

Em primeiro lugar, um sincero agradecimento ao professor Paulo Proença, cujo papel como orientador foi fundamental. A sua disponibilidade para rever o relatório inúmeras vezes permitiu-me apresentar uma versão mais refinada e estruturada deste trabalho.

Agradeço também ao David Mota, supervisor dos estágios na Devscope, a quem tive o prazer de conhecer na edição de \textit{FallStack} 2024. Foi ele quem me selecionou para este estágio e acolheu os meus interesses em áreas específicas da informática, que mais tarde se integrariam na minha proposta de estágio.

Um agradecimento especial ao meu \textit{buddy}, André Reis, pela orientação técnica e apoio especializado ao longo do projeto.

A todos os estagiários da Devscope, expresso a minha gratidão por tornarem esta experiência mais acolhedora e enriquecedora.

Quero também agradecer aos meus colegas de universidade, Rita Barbosa, Ana Guterres e Afonso Santos, que sempre me incentivaram e apoiaram durante a licenciatura. A sua amizade e motivação foram essenciais para o meu desenvolvimento académico e profissional.

Um agradecimento à Direção Geral de Ensino Superior, pelo apoio financeiro concedido através da bolsa de estudo durante os três anos da licenciatura, e à Câmara Municipal de Portimão, pelo apoio adicional durante dois anos. O contributo destas instituições foi crucial para que eu pudesse prosseguir os meus estudos.

Por fim, e com um carinho especial, quero agradecer aos meus pais, que, apesar dos desafios e dificuldades, sempre se esforçaram para que eu tivesse acesso ao ensino superior. Agradeço também ao meu namorado e à sua família, que têm sido um pilar fundamental de apoio para esta menina deslocada de casa.
\end{acknowledgements}

%----------------------------------------------------------------------------------------
%	ABSTRACT PAGE
%----------------------------------------------------------------------------------------

\chapter*{\textbf{Resumo}}


Os temas ESG (\textit{Environment}, \textit{Social} e \textit{Governance}) têm vindo a ganhar relevância nos últimos vinte anos, sobretudo no que respeita à divulgação de dados empresariais associados a estes domínios e às suas implicações. A questão do \textit{greenwashing}, as discrepâncias nas classificações ESG fornecidas por várias agências, as abordagens para reduzir o impacto das empresas nestas dimensões e os avanços tecnológicos associados são, atualmente, temas de grande relevância. Este projeto visou a criação de um protótipo de aplicação que permite monitorizar o desempenho ESG de uma empresa.

O presente relatório, elaborado no âmbito da unidade curricular Projeto/Estágio do curso de Licenciatura em Engenharia Informática do Instituto Superior de Engenharia do Porto, apresenta o processo de desenvolvimento da plataforma ESG, realizado para a empresa Devscope, entidade responsável pela orientação do projeto. A ferramenta permite que as organizações carreguem conjuntos de dados, avaliem o desempenho ao nível ESG de várias formas, façam corresponder métricas específicas ao seu contexto, misturem métricas com base na estrutura SASB e estabeleçam objetivos em relação às mesmas métricas.

O desenvolvimento do projeto baseou-se em várias etapas. A primeira fase consistiu na análise do estado da arte, na qual se estuda o conceito de ESG e as suas implicações. Seguiu-se o processo de engenharia de \textit{software}, na qual se procedeu à recolha e análise de requisitos, tanto funcionais como não funcionais. Com os requisitos delineados, procedeu-se à conceção do sistema, definindo a sua arquitetura com base nos modelos C4 e 4+1, bem como as suas funções. Por fim, foram efetuadas as fases de implementação e teste, bem como as evidências e a avaliação geral do que foi feito.

A solução criada cumpre os requisitos e objectivos definidos e é uma aplicação funciona como um \textit{hub} centralizado de apoio à gestão e visualização de temas aplicáveis à ESG para a Devscope. \\

\textbf{Palavras-chave (Tema):}	ESG (\textit{Environment, Social, Governance}), Métricas, Matriz de Materialidade, SASB (\textit{Sustainability Accounting Standards Board}), Conjuntos de Dados \\

\textbf{Palavras-chave (Tecnologias):} React, Next.js, ApexCharts, Railway, AWS S3, Jest

\vspace{20mm}


\begin{abstractotherlanguage}

ESG issues (Environment, Social and Governance) have become increasingly important over the last twenty years, especially with regard to the disclosure of corporate data associated with these areas and their implications. The issue of greenwashing, discrepancies in ESG classifications provided by various agencies, approaches to reducing the impact of companies in these dimensions and the associated technological advances are currently highly relevant topics. This project aimed to create a prototype application to monitor a company's ESG performance.

This report, produced as part of the Project/Internship course of the Degree in Computer Engineering at the Instituto Superior de Engenharia do Porto, presents the process of developing the ESG platform for the company Devscope, the organisation responsible for guiding the project. The tool allows organisations to upload data sets, assess performance at ESG level in various ways, match specific metrics to their context, mix metrics based on the SASB framework and set objectives in relation to the same metrics.

The project was developed in several stages. The first stage consisted of analysing the state of the art, studying the concept of ESG and its implications. This was followed by the software engineering process, in which both functional and non-functional requirements were collected and analysed. Once the requirements had been outlined, the system was designed, defining its architecture based on the C4 and 4+1 models, as well as its functions. Finally, the implementation and testing phases were carried out, as well as the evidence and general evaluation of what had been done.

The solution created meets the requirements and objectives defined and is an application that functions as a centralised hub to support the management and visualisation of ESG issues for Devscope. \\

\textbf{Keywords (Theme):}	ESG (Environment, Social, Governance), Metrics, Materiality Matrix, SASB (Sustainability Accounting Standards Board), Data Sets \\

\textbf{Keywords (Technologies):} React, Next.js, ApexCharts, Railway, AWS S3, Jest

\end{abstractotherlanguage}


%----------------------------------------------------------------------------------------
%	LIST OF CONTENTS/FIGURES/TABLES PAGES
%----------------------------------------------------------------------------------------

\tableofcontents % Prints the main table of contents

\listoffigures % Prints the list of figures

\listoftables % Prints the list of tables

% \iflanguage{portuguese}{
% \renewcommand{\listalgorithmname}{Lista de Algor\'itmos}
% }
% \listofalgorithms % Prints the list of algorithms
% \addchaptertocentry{\listalgorithmname}


\renewcommand{\lstlistlistingname}{List of Source Code}
\iflanguage{portuguese}{
\renewcommand{\lstlistlistingname}{Lista de C\'odigo}
}

\lstlistoflistings
\addcontentsline{toc}{chapter}{\lstlistlistingname}


%----------------------------------------------------------------------------------------
%	ABBREVIATIONS
%----------------------------------------------------------------------------------------
% \begin{abbreviations}{ll} % Include a list of abbreviations (a table of two columns)
% \begin{table}[h]
%     \centering
%     \begin{tabular}{|c|c|}
%         \hline
%         \textbf{ESG} & \textbf{E}nvironmental, \textbf{S}ocial and \textbf{G}overnance \\ \hline
%         \textbf{SASB} & \textbf{S}ustainability, \textbf{A}ccounting \textbf{S}tandards \textbf{B}oard \\ \hline
%         \textbf{GRI} & \textbf{G}lobal, \textbf{R}eporting \textbf{I}nitiative \\ \hline
%     \end{tabular}
% \end{table}
% \end{abbreviations}

%----------------------------------------------------------------------------------------
%	SYMBOLS
%----------------------------------------------------------------------------------------

% \begin{symbols}{lll} % Include a list of Symbols (a three column table)

% $a$ & distance & \si{\meter} \\
% $P$ & power & \si{\watt} (\si{\joule\per\second}) \\
% %Symbol & Name & Unit \\

% \addlinespace % Gap to separate the Roman symbols from the Greek

% $\omega$ & angular frequency & \si{\radian} \\

% \end{symbols}

%----------------------------------------------------------------------------------------
%	ACRONYMS
%----------------------------------------------------------------------------------------

% \newcommand{\listacronymname}{List of Acronyms}
% \iflanguage{portuguese}{
% 	\renewcommand{\listacronymname}{Lista de Acr\'onimos}
% }

% %Use GLS
% \glsresetall
% \printglossary[title=\listacronymname,type=\acronymtype,style=long]
% \addcontentsline{toc}{chapter}{\listacronymname}


\begin{acronyms}{|p{0.3\textwidth}|p{0.65\textwidth}|}
\hline
\textbf{Sigla} & \textbf{Definição} \\ \hline
API & Application Programming Interface \\
CRS & Corporate Social Responsibility \\
ESG & Environmental, Social, and Governance \\
GEE & Gases de Efeito Estufa \\
GRI & Global Reporting Initiative \\
HTML & HyperText Markup Language \\
HTTP & HyperText Transfer Protocol \\
IA & Inteligência Artificial \\
ISEP & Instituto Superior de Engenharia do Porto \\
IoT & Internet of Things \\
IT & Information Technology \\
JSON & JavaScript Object Notation \\
KPI & Key Performance Indicator \\
LEI & Licenciatura em Engenharia Informática \\
ODS & Objetivos de Desenvolvimento Sustentável \\
ONU & Organização das Nações Unidas \\
PESTI & Projecto / Estágio \\
PNL & Processamento de Linguagem Natural \\
PRI & Princípios para Investimento Responsável \\
RFID & Radio-Frequency Identification \\
SaaS & Software as a Service \\
SASB & Sustainability Accounting Standards Board \\
SPA & Single Page Application \\
UI & User Interface \\
UNEP FI & UNEP Finance Initiative \\
URL & Uniform Resource Locator \\
\hline
\end{acronyms}


%----------------------------------------------------------------------------------------
%	DONE
%----------------------------------------------------------------------------------------

\mainmatter % Begin numeric (1,2,3...) page numbering
\pagestyle{thesis} % Return the page headers back to the "thesis" style
