% we include the glossary here (frontmatter is included with \input, so this command is as if it was in main.tex)
\newacronym{ESG}{ESG}{Environmental, Social, and Governance }
\newacronym{GRI}{GRI}{Global Reporting Initiative}
\newacronym{SASB}{SASB}{Sustainability Accounting Standards Board}
\newacronym{ONU}{ONU}{Organização das Nações Unidas}
\newacronym{PESTI}{PESTI}{Projecto / Estágio}
\newacronym{PRI}{PRI}{Princípios para Investimento Responsável}
\newacronym{CRS}{CRS}{Corporate Social Responsibility}
\newacronym{PNL}{PNL}{Processamento de Linguagem Natural}
\newacronym{IA}{IA}{Inteligência Artificial}
\newacronym{ODS}{ODS}{Objetivos de Desenvolvimento Sustentável}
\newacronym{UNEP FI}{UNEP FI}{UNEP Finance Initiative}

\frontmatter % Use roman page numbering style (i, ii, iii, iv...) for the pre-content pages

\pagestyle{plain} % Default to the plain heading style until the thesis style is called for the body content

%----------------------------------------------------------------------------------------
%	TITLE PAGE
%----------------------------------------------------------------------------------------

% \maketitlepage


%----------------------------------------------------------------------------------------
%	STATEMENT of INTEGRITY
%----------------------------------------------------------------------------------------
\integritystatement

%----------------------------------------------------------------------------------------
%	DEDICATION  (optional)
%----------------------------------------------------------------------------------------
%
\begin{dedicatory}
Dedico este projeto a todos que me permitiram chegar a este capítulo da minha vida académica.
\end{dedicatory}

%----------------------------------------------------------------------------------------
%	ABSTRACT PAGE
%----------------------------------------------------------------------------------------

\begin{abstract}

% here you put the abstract in the main language of the work.

O resumo do relatório (que só deve ser escrito após o texto principal do relatório estar completo) é uma apresentação abreviada e precisa do trabalho, sem acrescento de interpretação ou crítica, escrita de forma impessoal, podendo ter, por exemplo, as seguintes três partes:

\begin{enumerate}
    \item Um parágrafo inicial de introdução do contexto e do problema/objetivo do trabalho.
    \item Resumo dos aspetos mais importantes do trabalho descrito no presente relatório, que por sua vez documenta abordagem adotada e sistematiza os aspetos relevantes do trabalho realizado durante o estágio. Deve mencionar tudo o que foi feito, por isso deve concentrar-se no que é realmente importante e ajudar o leitor a decidir se quer ou não consultar o restante do relatório.
    \item Um parágrafo final com as conclusões do trabalho realizado.
\end{enumerate}

Palavras-chave (Tema):	Incluir 3 a 6 palavras/expressões chave que caraterizem o projeto do ponto de vista de tema/área de intervenção.

Palavras-chave (Tecnologias): 	Incluir 3 a 6 palavras/expressões chave que caraterizem o projeto do ponto de vista de tecnologias utilizadas.

\vspace{20mm} %5mm vertical space

\textbf{(O Resumo só deve ocupar 1 página, cerca de 20 linhas.)}

\end{abstract}

\begin{abstractotherlanguage}

Here you put the abstract in the "other language": English, if the work is written in Portuguese; Portuguese, if the work is written in English.

\end{abstractotherlanguage}

%----------------------------------------------------------------------------------------
%	ACKNOWLEDGEMENTS (optional)
%----------------------------------------------------------------------------------------

\begin{acknowledgements}
    Gostaria de iniciar esta secção do relatório expressando a minha profunda gratidão às pessoas que me acompanharam ao longo deste percurso e contribuíram para a conceção deste projeto.
    
    Em primeiro lugar, um sincero agradecimento ao professor Paulo Proença, cujo papel como orientador foi fundamental. A sua disponibilidade para rever o relatório inúmeras vezes permitiu-me apresentar uma versão mais refinada e estruturada deste trabalho.
    
    Agradeço também ao David Mota, supervisor dos estágios na Devscope, a quem tive o prazer de conhecer na edição de FallStack 2024. Foi ele quem me selecionou para este estágio e acolheu os meus interesses em áreas específicas da informática, que mais tarde se integrariam na minha proposta de estágio.

    Um agradecimento especial ao meu buddy, André Reis, pela orientação técnica e apoio especializado ao longo do projeto.
    
    A todos os estagiários da Devscope, expresso a minha gratidão por tornarem esta experiência mais acolhedora e enriquecedora.
    
    Quero também agradecer aos meus colegas de universidade, Rita Barbosa, Ana Guterres e Afonso Santos, que sempre me incentivaram e apoiaram durante a licenciatura. A sua amizade e motivação foram essenciais para o meu desenvolvimento académico e profissional.
    
    Um agradecimento à DGES, pelo apoio financeiro concedido através da bolsa de estudo durante os três anos da licenciatura, e à Câmara Municipal de Portimão, pelo apoio adicional durante dois anos. O contributo destas instituições foi crucial para que eu pudesse prosseguir os meus estudos.
    
    Por fim, e com um carinho especial, quero agradecer aos meus pais, que, apesar dos desafios e dificuldades, sempre se esforçaram para que eu tivesse acesso ao ensino superior. Agradeço também ao meu namorado e à sua família, que têm sido um pilar fundamental de apoio para esta menina deslocada de casa.

\end{acknowledgements}


%----------------------------------------------------------------------------------------
%	LIST OF CONTENTS/FIGURES/TABLES PAGES
%----------------------------------------------------------------------------------------

\tableofcontents % Prints the main table of contents

\listoffigures % Prints the list of figures

\listoftables % Prints the list of tables

\iflanguage{portuguese}{
\renewcommand{\listalgorithmname}{Lista de Algor\'itmos}
}
\listofalgorithms % Prints the list of algorithms
\addchaptertocentry{\listalgorithmname}


\renewcommand{\lstlistlistingname}{List of Source Code}
\iflanguage{portuguese}{
\renewcommand{\lstlistlistingname}{Lista de C\'odigo}
}
\lstlistoflistings % Prints the list of listings (programming language source code)
\addchaptertocentry{\lstlistlistingname}


%----------------------------------------------------------------------------------------
%	ABBREVIATIONS
%----------------------------------------------------------------------------------------
% \begin{abbreviations}{ll} % Include a list of abbreviations (a table of two columns)
% \begin{table}[h]
%     \centering
%     \begin{tabular}{|c|c|}
%         \hline
%         \textbf{ESG} & \textbf{E}nvironmental, \textbf{S}ocial and \textbf{G}overnance \\ \hline
%         \textbf{SASB} & \textbf{S}ustainability, \textbf{A}ccounting \textbf{S}tandards \textbf{B}oard \\ \hline
%         \textbf{GRI} & \textbf{G}lobal, \textbf{R}eporting \textbf{I}nitiative \\ \hline
%     \end{tabular}
% \end{table}
% \end{abbreviations}

%----------------------------------------------------------------------------------------
%	SYMBOLS
%----------------------------------------------------------------------------------------

% \begin{symbols}{lll} % Include a list of Symbols (a three column table)

% $a$ & distance & \si{\meter} \\
% $P$ & power & \si{\watt} (\si{\joule\per\second}) \\
% %Symbol & Name & Unit \\

% \addlinespace % Gap to separate the Roman symbols from the Greek

% $\omega$ & angular frequency & \si{\radian} \\

% \end{symbols}

%----------------------------------------------------------------------------------------
%	ACRONYMS
%----------------------------------------------------------------------------------------

\newcommand{\listacronymname}{List of Acronyms}
\iflanguage{portuguese}{
\renewcommand{\listacronymname}{Lista de Acr\'onimos}
}

%Use GLS
\glsresetall
\printglossary[title=\listacronymname,type=\acronymtype,style=long]


%----------------------------------------------------------------------------------------
%	DONE
%----------------------------------------------------------------------------------------

\mainmatter % Begin numeric (1,2,3...) page numbering
\pagestyle{thesis} % Return the page headers back to the "thesis" style
