% 
\chapter{Implementação da Solução} 
\label{chap:Impl} % For referencing the chapter elsewhere, use Chapter~\ref{Impl}

Este capítulo deve ser dedicado à apresentação de detalhes relacionados com o enquadramento e implementação das soluções preconizadas no capítulo anterior. Note-se, no entanto, que detalhes desnecessários à compreensão do trabalho devem ser remetidos para anexos. Dependendo do volume, a avaliação do trabalho pode ser incluída neste capítulo ou pode constituir um capítulo separado.

%-------------------------------------------------------------------------------

\section{Descrição da implementação} 
\label{sec:desc} %For referencing this section elsewhere, use Section~\ref{sec:desc}

Descrever a implementação da solução proposta no capítulo anterior, podendo ser dadas explicações e evidências de soluções intercalares. Devem também ser descritas as tecnologias e metodologias utilizadas (software, sistemas de operação, linguagens, dispositivos ou outras ferramentas) e perspetiva crítica sobre as mesmas. 

Esta secção descreve a implementação da solução proposta no capítulo anterior. 

Alguns dos diagramas referidos na secção anterior podem aparecer nesta secção. Por exemplo, diagramas de classes ou diagramas de módulos, sendo detalhadas as operações de cada classe ou as funções de cada módulo (diagramas de atividades). 

Devem também ser descritas as especificidades de implementação de acordo com o ambiente de desenvolvimento, plataforma e linguagem escolhida para o desenvolvimento e deve ser claro que o desenho apresentado anteriormente foi, de facto, adotado na implementação

\textbf{Descrição da fase de implementação no processo de Engenharia de Software}

\subsection{Tecnologias Usadas}

\textbf{Menção da divisão das tecnologias em duas categorias: pré-desenvolvimento e desenvolvimento}

\subsubsection{Tecnologias de Pré-desenvolvimento}

- Plantuml \\
- VSCode \\
- Draw.io \\
- Figma (anexo com imagens dos mockups) \\
- Github Projects (mostrar setup do projecto e dos cartões) + Repository \\

Site do \textit{Figma}: \url{https://www.figma.com/} \\
Site do \textit{Github}: \url{https://github.com/} \\
Site do \textit{Draw\.io}: \url{https://www.drawio.com/} \\
Site do \textit{Plantuml}: \url{https://plantuml.com/} \\
Site do \textit{Visual Studio Code}: \url{https://code.visualstudio.com/} \\


\subsubsection{Tecnologias de Desenvolvimento}

- Visual Studio Code \\
- React \\
- Next.js \\
- Tailwind CSS \\
- http requests \\
- Vuexy template \\
- AWS S3 \\
- Prisma
- Railway com MySQL database \\


Site do \textit{React}: \url{https://react.dev/} \\
Site do \textit{Next.js}: \url{https://nextjs.org/} \\
Site do \textit{Tailwind CSS}: \url{https://tailwindcss.com/} \\
Site do \textit{Vuexy}: \url{https://pixinvent.com/vuexy-bootstrap-html-admin-template/} \\
Site do \textit{AWS S3}: \url{https://aws.amazon.com/s3/} \\
Site do \textit{Prisma}: \url{https://www.prisma.io/} \\
Site do \textit{Railway}: \url{https://railway.com/} \\


\section{Funcionalidades}

\subsection{Home Page | \textit{Dashboard}}

- KPIs \\
- ESG Score (+ formula) \\
- Best metrics since last record \\
- Worse metrics since last record \\

\subsection{Página das Métricas}

- Display \\
- Cards \\
- Filtering \\
- Creation of Metrics \\

\subsection{Mapa de Materialidade}

- Mapa Display \\
metion that when clicked it will open a generic metric card

\subsection{Página dos Datasets}

- Display \\
- Card \\
- Show Filtering \\
- Show import \\
- Show deletion (maybe) \\

\subsection{Página dos Objetivos}

mention its a complementary part tha the company showed interest in implementing

- Display \\
- Creation of objectives \\

\section{Testes} 

- Teste unitarios aos componentes
- Alguns de integraçao

mostrar apenas print das cenas a passarem + 1 exemplo de unitario, 1 exemplo de integraçao

(maybe E2E ???)

A descrição dos testes efetuados (e.g. unitários, funcionais, de integração, de sistema) pode ser feita nesta secção ou, caso não se justifique, na secção anterior.

\section{Avaliação da solução} 

Nesta secção deve ser descrita a abordagem seguida para avaliar a solução, ou parte da solução, nomeadamente um ou mais requisitos de qualidade (e.g. desempenho, usabilidade). 

São descritas experiências efetuadas e apresentados os dados/modelos utilizados, bem como os resultados obtidos. 

Devem ser descritos e avaliados os resultados obtidos. 

Deve ser feita uma discussão sobre a adequação dos resultados obtidos relativamente aos planeados anteriormente. 

Esta secção poderá não existir em alguns relatórios de projeto/estágio, mas nesse caso deverá ser dada uma justificação para tal. 
