\chapter{Uso de Inteligência Artificial}
\label{chap:AI}

Neste Capítulo, procura-se esclarecer, de forma transparente, a utilização da Inteligência Artificial (IA) durante a elaboração deste documento. A aplicação desta ferramenta foi restrita, esporádica e sempre em busca de dar clareza e coerência ao texto, sem comprometer o rigor analítico ou a autoria do texto.

As áreas em que foram utilizadas a inteligência artificial foram:

\begin{itemize}
  \item As Tabelas~\ref{tab:gri_sasb}, \ref{tab:comparacao_solucoes_esg} e \ref{tab:comparacao_frontend}: toda a análise, recolha de informação e conclusões foram feitas manualmente. A inteligência artificial foi utilizada apenas para reorganizar o texto, melhorar a coesão dos resumos e facilitar a adaptação ao espaço disponível nas tabelas, sem alterar o conteúdo fundamental.

  \item A Secção~\ref{subsec:FunReq}, aborda a utilização de inteligência artificial na normalização da escrita dos requisitos funcionais. A informação e as especificações de cada requisito foram determinadas antecipadamente, sendo a inteligência artificial utilizada apenas para garantir que existia um formato e estilo uniformes nos vários casos de utilização.
\end{itemize}
