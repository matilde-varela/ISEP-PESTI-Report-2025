\chapter{Conclusões} 
\label{chap:Conc}

Neste último Capítulo, é apresentada uma síntese do trabalho efetuado, bem como os resultados obtidos, que sustentam as conclusões finais. Os objetivos definidos para o projeto são reafirmados e é discutida a forma como os resultados os cumprem. De seguida, são apontadas as principais limitações encontradas, bem como recomendações para futuras extensões e melhorias da solução desenvolvida, seguindo-se uma análise global do projeto.

\section{Objetivos concretizados} 
\label{sec:ObjConc}

Tal como referido no Capítulo \ref{chap:Intro}, os objetivos especificados e o grau de concretização desses objetivos estão resumidos na Tabela \ref{tab:objective_done_table}:

\begin{table}[H]
    \rowcolors{2}{purple!10}{white}
    \renewcommand{\arraystretch}{1.2}
    \setlength{\tabcolsep}{10pt}
    \centering
    \begin{tabular}{>{\bfseries}p{11cm} p{4cm}}
        \rowcolor{purple!40}
        \textbf{Objetivo} & \textbf{Realização} \\
        \textbf{Consolidar dados de múltiplas fontes} & Realizado \\
        \textbf{Desenvolver uma aplicação configurável} & Realizado \\
        \textbf{Visualização clara e interactiva dos dados ESG} & Realizado \\
        \textbf{Definir e monitorizar objetivos baseados em métricas} & Realizado \\
    \end{tabular}
    \caption{Objetivos do Projeto e o seu Grau de Realização}
    \label{tab:objective_done_table}
\end{table}

Todos os objetivos estabelecidos foram plenamente atingidos. A solução criada permite a centralização da informação ESG, a definição de objetivos baseados em metas ligadas a métricas (sejam estas pré-existentes ou personalizadas), tem uma interface adaptável e oferece uma visualização gráfica visível e interactiva dos dados.

A planificação cuidadosa e análise prévia permitiram uma implementação eficaz e livre de complicações, garantindo a realização dos objetivos definidos.

\section{Trabalho Futuro e Limitações}

Apesar do facto de a solução proposta ser eficiente e completa, foram descobertas algumas limitações ao longo do processo de desenvolvimento. A funcionalidade do caso de uso UC-07 (Importação de Dados) tem limitações nos formatos de ficheiros suportados. De momento, apenas são suportados ficheiros CSV com uma estrutura predefinida.

Como consequência da limitação acima mencionada, a representação dos diagramas está igualmente limitada, pois a biblioteca usada, o \textit{ApexCharts} disponibiliza 20 diagramas diferentes, enquanto que a plataforma apenas usufrui de 5.

Como trabalho futuro, destaca-se a possibilidade de:

\begin{itemize}
\item Aumentar ou melhorar a importação para aceitar vários tipos de ficheiros e estruturas
\item Explorar todas as categorias de gráficos disponíveis na biblioteca \textit{ApexCharts}
\item Facilitar a intuitividade do processo de edição dos dados que foram importados 
\item Implementar a criação automática de relatório com fluxos de recolha, tratamento e apresentação de dados
\item Implementar um mecanismo de autenticação para limitar o acesso à plataforma apenas a utilizadores autorizados
\end{itemize}

Estas futuras extensões acrescentariam um valor significativo à solução e ao posicionamento da empresa no domínio ESG.

\section{Considerações Finais}

O planeamento e desenvolvimento do projeto, no âmbito profissional da DevScope, foram extremamente proveitoso. O ambiente positivo de troca de conhecimento, apoio e consistência contribuiu significativamente para a efetiva realização e implementação bem sucedida das funcionalidades propostas.

O projeto ofereceu a possibilidade de aplicação de competências aprendidas durante a formação universitária e, simultaneamente, de aprender novos conhecimentos, tanto através do trabalho desenvolvido quanto pelas várias formações e \textit{workshops} que foram proporcionados pela DevScope.

Revendo o percurso percorrido existem certos aspetos que, caso a proposta do problema apresentado fosse feita novamente, poderiam ser melhorados.

A propostas inicialmente apresentada para o estágio foi a criação de uma plataforma de apoio à avaliação ESG, com ênfase na vertente técnica de implementação e integração das respectivas funcionalidades, incluindo o acompanhamento das medidas. No entanto, no processo de desenvolvimento do projeto, tornou-se claro que a proposta inicial poderia ter sido mais precisa através de uma melhor definição dos objectivos estratégicos da aplicação, como por exemplo: o perfil do utilizador final, os fluxos de trabalho pretendidos e os requisitos organizacionais reais em termos de monitorização ESG. Uma melhor articulação inicial com a organização teria permitido um alinhamento mais cedo do foco da aplicação com as suas prioridades operacionais e tecnológicas.

Ao longo do desenvolvimento, utilizou-se o \textit{React.js} para o \textit{frontend}, aliado ao \textit{template Vuexy}, que permitiu a criação rápida de interfaces modernas.

Apesar disso, desde o início, teria sido viável adotar uma abordagem mais modular, com o objetivo de simplificar as manutenções futuras e a adição de novas funcionalidades, particularmente num contexto empresarial em que as equipas trabalham remotamente. No projeto foi utilizada uma base de dados \textit{MySQL} com um esquema relacional bem definido.

Apesar de eficaz, é possível observar que a utilização de um sistema de gestão de bases de dados como o \textit{PostgreSQL}, que tem suporte para campos dinâmicos, ou mesmo uma solução \textit{NoSQL}, poderia potencialmente proporcionar mais flexibilidade, especialmente no tratamento de grandes diferenças entre sectores ou empresas, normalmente observadas no caso das práticas ESG.

Do ponto de vista infraestrutural, foi configurado armazenamento na \textit{cloud} através do AWS S3. Esta escolha foi prática e facilitou o desenvolvimento local e a prototipagem, mas a utilização do \textit{Azure Blob Storage} teria sido mais integrada no ecossistema tecnológico que a empresa já possui, e a integração no futuro numa configuração real teria sido mais simples. Contudo, a escolha e exploração de uma tecnologia diferente ao ecossistema da empresa foi incentivada, por se tratar de uma prova de conceito e não de um projeto com vista a produção.

Ao nível da investigação e fundamentação, foi efectuada uma comparação aprofundada dos principais \textit{framework} ESG, o GRI e o SASB. A decisão de utilizar o SASB foi acertada, tendo em conta o facto de ser o padrão já utilizado pela empresa, mas a plataforma poderia ter sido concebida desde o início com mais abstração para suportar várias \textit{framework}, o que lhe daria maior escalabilidade e reutilização por outras empresas.

A investigação de ferramentas de mercado como a \textit{Workiva, IBM ESG Suite} e \textit{SAP Sustainability Control Tower} permitiu apoiar decisões funcionais, mas uma análise técnica mais aprofundada das suas API, fluxos de dados ou práticas de integração poderia ter enriquecido ainda mais a arquitetura da aplicação, nomeadamente ao nível da gestão de objectivos.

Por último, apesar de dispor de testes unitários e de integração, o projeto beneficiaria de uma \textit{pipeline} de integração contínua (CI/CD) mais robusta, como a execução automática de testes e implementações. Esta melhoria permitiria uma simulação mais realista do ambiente de produção e alinharia a prática de desenvolvimento com as práticas modernas de \textit{DevOps}. 

Em suma, a aplicação desenvolvida serve bastante bem os objectivos definidos. No entanto, a sua proposta e realização podem ser melhoradas se forem tomadas algumas decisões estratégicas e tecnológicas numa perspetiva mais coerente com o contexto empresarial, juntamente com as necessidades de escalabilidade e manutenção a longo prazo.

A nível pessoal, a experiência foi especialmente gratificante. O tema do projeto desperta um grande interesse por parte do autor, o que aumentou o envolvimento. Além disso, o desafio foi uma oportunidade de resiliência e crescimento, incluindo interações com profissionais experientes do setor.

A empresa forneceu um estímulo constante e \textit{feedback} pertinente, inclusive na redação do relatório.

Para concluir, o projeto foi uma experiência muito enriquecedora, tanto do ponto de vista técnico como humano, que abriu portas para futuras possibilidades profissionais e reforçou competências básicas para a integração no mercado de trabalho.