\chapter{Conclusões} 
\label{chap:Conc}

Neste último Capítulo, é apresentada uma síntese do trabalho efectuado, bem como os resultados obtidos, que sustentam as conclusões finais. Os objectivos definidos para o projeto são reafirmados e é discutida a forma como os resultados os cumprem. De seguida, são apontadas as principais limitações encontradas, bem como recomendações para futuras extensões e melhorias da solução desenvolvida, seguindo-se uma análise global do projeto.

\section{Objetivos concretizados} 
\label{sec:ObjConc}

Tal como referido no Capítulo \ref{chap:Intro}, os objectivos especificados e o grau de concretização desses objectivos estão resumidos na Tabela \ref{tab:objective_done_table}:

\begin{table}[H]
    \rowcolors{2}{purple!10}{white}
    \renewcommand{\arraystretch}{1.2}
    \setlength{\tabcolsep}{10pt}
    \centering
    \begin{tabular}{>{\bfseries}p{11cm} p{4cm}}
        \rowcolor{purple!40}
        \textbf{Objetivo} & \textbf{Realização} \\
        \textbf{Consolidar dados de múltiplas fontes} & Realizado \\
        \textbf{Desenvolver uma aplicação configurável} & Realizado \\
        \textbf{Visualização clara e interactiva dos dados ESG} & Realizado \\
        \textbf{Definir e monitorizar objectivos baseados em métricas} & Realizado \\
    \end{tabular}
    \caption{Objetivos do Projeto e o seu Grau de Realização}
    \label{tab:objective_done_table}
\end{table}

Todos os objectivos estabelecidos foram plenamente atingidos. A solução criada permite a centralização da informação ESG, a definição de objectivos baseados em metas ligadas a métricas (sejam estas pré-existentes ou personalizadas), tem uma interface adaptável e oferece uma visualização gráfica visível e interactiva dos dados.

A planificação cuidadosa e análise prévia permitiram uma implementação eficaz e livre de complicações, garantindo a realização dos objectivos definidos.

\section{Trabalho Futuro e Limitações}

Apesar do facto de a solução proposta ser eficiente e completa, foram descobertas algumas limitações ao longo do processo de desenvolvimento. A funcionalidade do caso de uso UC-07 (Importação de Dados) tem limitações nos formatos de ficheiros suportados. De momento, apenas são suportados ficheiros CSV com uma estrutura predefinida. Isto foi recomendado pelo \textit{buddy} do aluno, tendo em conta a complexidade envolvida e o tempo disponível para a implementação da funcionalidade.

Como consequência disso, a representação dos diagramas está igualmente limitada, pois a biblioteca usada, o \textit{ApexCharts} disponibiliza 20 diagramas diferentes, enquanto que a plataforma apenas usufrui de 5.

Como trabalho futuro, destaca-se a possibilidade de:

\begin{itemize}
\item Aumentar ou melhorar a importação para aceitar vários tipos de ficheiros
\item Explorar todas as categorias de gráficos disponíveis na biblioteca \textit{ApexCharts}
\item Facilitar a intuitividade do processo de edição dos dados que foram importados 
\item Implementar a criação automática de relatório com fluxos de recolha, tratamento e apresentação de dados
\item Implementar um mecanismo de autenticação para limitar o acesso à plataforma apenas a utilizadores autorizados
\end{itemize}

Estas futuras extensões acrescentariam um valor significativo à solução e ao posicionamento da empresa no domínio ESG.

\section{Considerações Finais}

O planeamento e desenvolvimento do projeto, no âmbito profissional da DevScope, foram extremamente proveitoso. O ambiente positivo de troca de conhecimento, apoio e consistência contribuiu significativamente para a efetiva realização e implementação bem sucedida das funcionalidades propostas.

O projeto ofereceu a possibilidade de aplicação de competências aprendidas durante a formação universitária e, simultaneamente, de aprender novos conhecimentos, tanto através do trabalho desenvolvido quanto pelas várias formações e \textit{workshops} que foram proporcionados pela DevScope.

A nível pessoal, a experiência foi especialmente gratificante. O tema do projeto desperta um grande interesse por parte do autor, o que aumentou o envolvimento. Além disso, o desafio foi uma oportunidade de resiliência e crescimento, incluindo interações com profissionais experientes do setor.

A empresa forneceu um estímulo constante e feedback pertinente, inclusive na redação do relatório.

Em suma, o projeto foi uma experiência muito enriquecedora, tanto do ponto de vista técnico como humano, que abriu portas para futuras possibilidades profissionais e reforçou competências básicas para a integração no mercado de trabalho.