% 
\chapter{Análise e desenho da solução} % Main chapter title
\label{chap:ADS} % For referencing the chapter elsewhere, use Chapter~\ref{ADS}

O estudante deve questionar se o relatório descreve o trabalho de forma suficientemente detalhada para que possa ser compreendido e reproduzido, se necessário, no futuro por outras pessoas da organização. 

Todas as boas práticas abordadas no curso deverão ser utilizadas neste capítulo e no relatório em geral. 

Note-se que este capítulo não deve conter exclusivamente a explicação da forma como métodos, técnicas, algoritmos, tecnologias, etc. foram usados, mas também o processo de compreensão do problema, a análise nos seus vários níveis, a identificação e especificação de requisitos, a modelação, a descrição dos componentes da solução, etc. 

Recomenda-se que a descrição técnica siga uma abordagem que parta do “geral” (descrição inicial do problema) para o “particular” (descrição técnica completa e coerente da solução), sem saltar etapas.


\textbf{Nota}: Pode justificar-se este capítulo ser divido em dois (Análise do Problema e Desenho da Solução).

\textbf{Atenção}: A proposta de estrutura das subsecções seguintes adequa-se a projetos/estágios de desenvolvimento de produto ou sistema. O estudante deve, conjuntamente com o orientador, definir a estrutura de secções mais adequada ao seu projeto.

%-------------------------------------------------------------------------------

\section{Domínio do problema} 
\label{sec:DP} 
Devem ser especificados os conceitos de domínio do problema através de artefactos adequados (e.g. glossário, modelo de domínio). 

\section{Requisitos funcionais e não funcionais} 
\label{sec:Req} 
Especificar os principais requisitos funcionais e não funcionais do sistema. 

O levantamento de requisitos pode ser obtido dialogando com o cliente, de forma a identificar as funcionalidades de alto nível pretendidas no sistema para cada perfil de utilizador, recorrendo a diagramas de casos de utilização e/ou diagramas de domínio. 

De um modo geral, pretende-se a documentação final dos requisitos (e não a sua evolução no tempo).

\section{Desenho} 
\label{sec:Des} 
Dependendo do volume, o desenho da solução pode ser incluída neste capítulo (uma secção) ou pode constituir um capítulo separado.

O estudante deve especificar a arquitetura global da solução (diagramas de componentes e outros se necessário, como por exemplo, diagramas de implantação caso existam instalações) e deve apresentar uma justificação para a arquitetura que faça referência aos requisitos. 
Se foram estudadas alternativas arquiteturais, estas devem ser apresentadas.

Apresentar uma especificação global do sistema com base em um ou mais diagramas de Classes e apresentar especificações adicionais em casos que o justifiquem (e.g. diagramas de estados para classes com comportamento relativamente complexo).

Especificar o modelo de dados, caso a solução contemple esse aspeto, e justificar o modelo de dados com base nos artefactos anteriores (particularmente o diagrama de classes).
A aplicação de padrões e boas práticas é recomendada.


