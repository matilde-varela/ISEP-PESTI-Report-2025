\chapter{Introdução}
\label{chap:Intro}

Este Capítulo tem como objetivo apresentar o projeto, o trabalho realizado e a sua contextualização, abordando o problema em estudo e os objetivos a serem alcançados. Também será feita uma descrição das principais metodologias utilizadas ao longo do trabalho, bem como a identificação dos contributos e dos aspetos inovadores da solução desenvolvida. O Capítulo conclui com a apresentação da estrutura do presente documento.

%-------------------------------------------------------------------------------

\section{Enquadramento/Contexto} 
\label{sec:chap1_introduction} 

Este documento descreve o trabalho realizado na unidade curricular de \gls{PESTI} da Licenciatura em Engenharia Informática (LEI) do \gls{ISEP} e constitui um dos elementos mais importantes do curso. Através dele, é atribuído um problema a ser resolvido, esperando-se uma solução documentada neste relatório. O principal objetivo deste projeto é desenvolver competências pessoais e profissionais, aplicando os conhecimentos adquiridos ao longo da licenciatura e preparando o estudante para a inserção no mercado de trabalho.

O presente projeto foi desenvolvido durante um estágio em ambiente empresarial na DevScope, uma empresa sediada no Porto, fundada em 2003, com mais de vinte anos de experiência em consultoria e desenvolvimento de \textit{software}. A empresa especializa-se em tecnologias da Microsoft, nomeadamente \textit{Power Platform}, \textit{Portals} (Office 365 \& SharePoint), \textit{Web \& App Development}, \textit{AI \& Machine Learning}, \textit{Business Intelligence}, \textit{Enterprise Integration}, \textit{Cloud \& DevOps} e \textit{Training \& Education} (\cite{DevScopeSolutions}). Os seus serviços são aplicados em diversas áreas, incluindo saúde, retalho e setor imobiliário.

Nos últimos anos, as empresas têm demonstrado uma preocupação crescente com questões dos tipos \textit{\gls{ESG}}. Este foco tem levado as organizações a recolher e analisar dados que auxiliem numa tomada de decisão mais consciente e estratégica. No entanto, a falta de centralização no tratamento e visualização desses dados pode comprometer a performance empresarial e expô-la a riscos, como falhas de conformidade regulatória ou problemas éticos.

Assim, torna-se fundamental dispor de uma plataforma que compile e organize estes dados de forma acessível e estruturada, permitindo às empresas obter \textit{insights} mais claros e estratégicos sobre as suas iniciativas \gls{ESG}.

O presente projeto tem como objetivo o desenvolvimento do \textit{frontend} desta plataforma, tornando-se o primeiro passo para a criação de uma ferramenta interna da DevScope. Além de contribuir para o desenvolvimento de um protótipo funcional, o projeto permite ao estudante aprender e aplicar ferramentas e tecnologias não abordadas no currículo académico, enriquecendo assim as suas competências técnicas e práticas.


\section{Descrição do Problema}

Nos últimos anos, a DevScope tem vindo a crescer exponencialmente, passando de uma empresa com um ambiente mais familiar para uma estrutura de maior dimensão. Com esse crescimento, surgiu a necessidade de uma plataforma que permita monitorizar e gerir de forma estruturada as iniciativas \gls{ESG}.

Atualmente, a empresa enfrenta desafios na centralização e análise de métricas \gls{ESG}, dificultando a identificação de padrões e a tomada de decisões informadas. Questões como consumo de eletricidade, temperatura do escritório e outras métricas ambientais, assim como indicadores sociais e de governança, precisam de ser monitorizados de forma mais acessível e eficiente.

A implementação desta plataforma permitirá um acompanhamento mais estruturado das métricas \gls{ESG}, facilitando auditorias, identificando comportamentos que possam gerar custos desnecessários e garantindo uma maior transparência. Sem uma solução eficaz, a empresa pode enfrentar perdas financeiras devido a desperdícios operacionais, além de possíveis riscos reputacionais e regulatórios caso não consiga demonstrar conformidade com boas práticas \gls{ESG}.

O projeto visa desenvolver a interface do utilizador para esta plataforma, garantindo uma experiência intuitiva e eficiente na visualização e gestão dos dados \gls{ESG}, ao utilizar \textit{frameworks} reconhecidas como a \textit{\gls{GRI}} ou \textit{Sustainability Accounting Standards Board (SASB)} para a definição das métricas.


\subsection{Objetivos}

O foco do projeto é o desenvolvimento de um protótipo \textit{frontend} de uma ferramenta interna da DevScope com o objetivo de rastrear e avaliar métricas \gls{ESG} para levar a uma tomada de decisão mais informada, em alinhamento com as necessidades específicas de cada organização.

A nível técnico, o projeto tem como objetivos:

\begin{itemize}
\item \textbf{Consolidar dados de múltiplas fontes:} Combinar conjuntos relevantes de dados no mesmo sistema, fornecendo uma visão unificada e acessível.
\item \textbf{Desenvolver uma aplicação configurável:} Permitir que as organizações configurem a ferramenta de acordo com as suas prioridades, sectores e objetivos estratégicos.
\item \textbf{Visualização clara e interactiva dos dados ESG:} Conceber uma interface intuitiva que simplifique a análise e interpretação dos dados relacionados com ESG.
\item \textbf{Definir e monitorizar objetivos baseados em métricas:} Facilitar o estabelecimento de objetivos personalizados de acordo com a informação ESG disponível, e simplificar a monitorização ao longo do tempo.
\end{itemize}

\subsection{Abordagem} 

O desenvolvimento do projeto seguiu uma abordagem ágil, caracterizada pela iteratividade do \textit{software} desenvolvido, pela compreensão das prioridades e pela capacidade de introduzir mudanças ao longo do processo. A flexibilidade perante os pedidos do cliente (DevScope) foi um dos principais fatores, promovendo uma comunicação próxima e frequente (\cite{Patel2025}). O projeto fará ainda uso da \textit{agile framework Scrum}, sendo o projeto divido em iterações de uma semana (\textit{sprints}) e diferentes fases, tais como: pré-planeamento do \textit{sprint} (\textit{release backlog} e \textit{sprint goals}), planeamento do \textit{sprint}/iteração (\textit{sprint backlog}), implementação e demonstração das funcionalidades desenvolvidas (\cite{Cohen2004}).

Além das metodologias mencionadas, foi utilizado o \textit{Microsoft OneNote}, uma ferramenta adotada pela DevScope como base de conhecimento colaborativa, desenvolvida pelos próprios colaboradores e acessível a toda a empresa. O \textit{GitHub} foi empregado não apenas como repositório da solução desenvolvida, mas também como \textit{hub} central do projeto, integrando o quadro \textit{Kanban}. Esta abordagem, alinhada às metodologias ágeis, visa minimizar o tempo ocioso, promover um fluxo de trabalho mais eficiente e garantir um gerenciamento de tarefas estruturado, por meio de cartões detalhados que acompanham o progresso de cada etapa (\cite{Wakode2021Kanban}).

As funções de \textit{product owner} e \textit{scrum master} foram exercidas por supervisores da empresa.

O estagiário teve flexibilidade horária, mas manteve comunicação contínua com a empresa. As reuniões diárias (\textit{daily stand-ups}, ~30 min) serviram para atualizar a equipa, enquanto as reuniões semanais garantiram acompanhamento mais detalhado com o supervisor, incluindo demonstrações e \textit{feedback} para orientar o desenvolvimento.

\subsection{Contributos}

A solução desenvolvida oferece à DevScope uma ferramenta que centraliza e processa informação importante e permite todo um conjunto de benefícios, contribuindo para a mesma de diversas formas:
\begin{itemize}
  \item \textbf{Aumento da produtividade:} Facilitar o acesso e análise das métricas ESG, agilizando o processo de tomada de decisões.
  \item \textbf{Tomadas de decisão mais informadas e estruturadas:} Fornecer dados confiáveis e organizados que suportem decisões empresariais baseadas em métricas claras.
  \item \textbf{Economia financeira:} Identificar comportamentos ineficientes que possam gerar custos desnecessários, promovendo maior eficiência.
  \item \textbf{Avaliação contínua de comportamentos empresariais:} Permitir a monitorização constante de práticas ESG, incentivando um comportamento mais responsável e sustentável.
\end{itemize}

Para o estudante, a solução desenvolvida representa o culminar de uma experiência enriquecedora, que exigiu a aquisição e o aprimoramento de novas competências. Além disso, proporcionou a sua integração num ambiente empresarial real, com as suas regras, processos, políticas e exposição à cultura organizacional.

\subsection{Planeamento do trabalho}

O projeto iniciou-se com uma fase de \textit{Ramp Up} de três semanas, durante a qual são realizados \textit{workshops} sobre as tecnologias utilizadas na DevScope. Após essa etapa inicial, o estágio seguiu um cronograma definido pela empresa, dividido em três fases principais: pesquisa e estado da arte, implementação da solução e refinamento do relatório e da solução. A escrita deste documento teve início após o período de \textit{Ramp Up}.

O cronograma do estágio foi organizado da seguinte forma:

\begin{itemize}
    \item \textbf{Ramp Up}: 24 de fevereiro a 14 de março
    \item \textbf{Estado da Arte}: 17 de março a 4 de abril
    \item \textbf{Implementação}: 14 de abril a 9 de maio
    \item \textbf{Refinamento e Revisão}: 19 de maio a 6 de junho
\end{itemize}

No Apêndice \ref{AppendixA}, apresenta-se um diagrama de \textit{Gantt} detalhado com o planeamento definido.

Devido ao aumento de complexidade do projeto desenvolvido, surgiram desvios do planeamento inicial, levando a que a submissão final fosse apontada para 6 de julho, de modo a concluir a implementação, testagem, escrita e refinamento do documento.

\section{Uso de Inteligência Artificial}

Nesta secção procura-se esclarecer, de forma transparente, a utilização da Inteligência Artificial (IA) durante a elaboração deste documento. A aplicação desta ferramenta foi restrita, esporádica e sempre em busca de dar clareza e coerência ao texto, sem comprometer o rigor analítico ou a autoria do texto.

As áreas em que foram utilizadas a inteligência artificial foram:

\begin{itemize}
  \item As Tabelas~\ref{tab:gri_sasb}, \ref{tab:comparacao_solucoes_esg} e \ref{tab:comparacao_frontend}: toda a análise, recolha de informação e conclusões foram feitas manualmente. A inteligência artificial foi utilizada apenas para reorganizar o texto, melhorar a coesão dos resumos e facilitar a adaptação ao espaço disponível nas tabelas, sem alterar o conteúdo fundamental.

  \item A Secção~\ref{subsec:FunReq}, aborda a utilização de inteligência artificial na normalização da escrita dos requisitos funcionais. A informação e as especificações de cada requisito foram determinadas antecipadamente, sendo a inteligência artificial utilizada apenas para garantir que existia um formato e estilo uniformes nos vários casos de utilização.
\end{itemize}


\section{Estrutura do relatório}

O presente documento é composto por seis capítulos que descrevem o desenvolvimento da solução pretendida.

A \textbf{Introdução} contextualiza o projeto, apresenta o problema e as abordagens adotadas, destaca os contributos para os \textit{stakeholders} e expõe o cronograma.

O \textbf{Estado da Arte} analisa trabalhos e tecnologias relevantes, explorando soluções existentes relacionadas com o problema em questão.

Os Capítulos de \textbf{Análise e Desenho da Solução} detalha a metodologia, as tecnologias e os algoritmos utilizados, além da modelação da solução e especificação de requisitos.

A \textbf{Implementação da Solução} descreve aspetos técnicos, como \textit{software}, ferramentas e sistemas utilizados, bem como os testes realizados e a avaliação da solução.

Por fim, as \textbf{Conclusões} apresentam um balanço dos resultados obtidos, analisam as limitações encontradas e incluem uma reflexão crítica sobre o trabalho desenvolvido.

\vspace{20mm} 
