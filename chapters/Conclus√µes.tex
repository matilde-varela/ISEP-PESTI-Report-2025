\chapter{Conclusões} 
\label{chap:Conc}

Neste último Capítulo, é apresentada uma síntese do trabalho efetuado, bem como os resultados obtidos, que sustentam as conclusões finais. Os objetivos definidos para o projeto são reafirmados e é discutida a forma como os resultados os cumprem. De seguida, são apontadas as principais limitações encontradas, bem como recomendações para futuras extensões e melhorias da solução desenvolvida, seguindo-se uma análise global do projeto.

\section{Objetivos concretizados} 
\label{sec:ObjConc}

Tal como referido no Capítulo \ref{chap:Intro}, os objetivos especificados e o estado de concretização desses objetivos estão resumidos na Tabela \ref{tab:objective_done_table}:

\begin{table}[H]
    \centering
    \begin{tabular}{|>{\bfseries}p{11cm} |p{4cm}|}
        \hline
        \textbf{Objetivo} & \textbf{Estado} \\
                \hline
        Consolidar dados de múltiplas fontes & Realizado \\
                \hline
        Desenvolver uma aplicação configurável & Realizado \\
                \hline
        Visualização clara e interactiva dos dados ESG & Realizado \\
                \hline
        Definir e monitorizar objetivos baseados em métricas & Realizado \\
                \hline
    \end{tabular}
    \caption{Objetivos do Projeto e o seu Estado de Concretização (autoria própria)}
    \label{tab:objective_done_table}
\end{table}

Todos os objetivos estabelecidos foram plenamente atingidos. A solução criada permite a centralização da informação ESG, a definição de objetivos baseados em metas ligadas a métricas (sejam estas pré-existentes ou personalizadas), tem uma interface adaptável e oferece uma visualização gráfica visível e interactiva dos dados.

A planificação cuidadosa e análise prévia permitiram uma implementação eficaz e livre de complicações, garantindo a realização dos objetivos definidos.

\section{Trabalho Futuro e Limitações}

Apesar do facto de a solução proposta ser eficiente e completa, foram descobertas algumas limitações ao longo do processo de desenvolvimento. A funcionalidade do caso de uso UC-07 (Importação de Dados) tem limitações nos formatos de ficheiros suportados. De momento, apenas são suportados ficheiros CSV com uma estrutura predefinida.

Como consequência da limitação acima mencionada, a representação dos diagramas está igualmente limitada, pois a biblioteca usada, o \textit{ApexCharts} disponibiliza 20 diagramas diferentes, enquanto que a plataforma apenas usufrui de 5.

Por último, apesar de dispor de testes unitários e de integração, o projeto beneficiaria de uma \textit{pipeline} de integração contínua (CI/CD) mais robusta, como a execução automática de testes e implementações. Esta melhoria permitiria uma simulação mais realista do ambiente de produção e alinharia a prática de desenvolvimento com as práticas modernas de \textit{DevOps}. 

Como trabalho futuro, destaca-se a possibilidade de:

\begin{itemize}
\item Aumentar ou melhorar a importação para aceitar vários tipos de ficheiros e estruturas
\item Explorar todas as categorias de gráficos disponíveis na biblioteca \textit{ApexCharts}
\item Facilitar a intuitividade do processo de edição dos dados que foram importados 
\item Implementar a criação automática de relatório com fluxos de recolha, tratamento e apresentação de dados
\item Implementar um mecanismo de autenticação para limitar o acesso à plataforma apenas a utilizadores autorizados
\item Impôr uma \textit{pipeline} de integração contínua (CI/CD), com a execução automática de testes e implementações
\end{itemize}

Estas futuras extensões acrescentariam um valor significativo à solução e ao posicionamento da empresa no domínio ESG.

\section{Considerações Finais}

A presente Secção divide-se em duas partes: uma perspetiva crítica do problema apresentado pela empresa e das implicações no desenvolvimento da solução, e, por fim, considerações pessoais perante a experiência de estágio e o trabalho desenvolvido.

\subsection{Perspetiva Crítica do Problema definido e Solução proposta}

O projeto permitiu a utilização das capacidades adquiridas através da formação académica, desenvolvendo simultaneamente novas competências através do desenvolvimento prático e de vários \textit{workshops} promovidos pela DevScope.

A propostas inicialmente apresentada foi a criação de uma plataforma de apoio à avaliação ESG, com ênfase na vertente técnica de implementação e integração das respectivas funcionalidades, incluindo o acompanhamento e gestão das medidas. 

Porém, caso se trate de um projeto de produção, os objetivos estratégicos poderiam ter sido formulados de forma mais clara em termos do perfil do utilizador final, dos fluxos de trabalho pretendidos e dos requisitos organizacionais reais em termos de monitorização ESG. Uma melhor articulação inicial com a organização teria permitido um alinhamento mais precoce no foco da aplicação com as suas prioridades operacionais e tecnológicas. 

Durante o desenvolvimento, a utilização do \textit{React.js} para o \textit{frontend}, em conjunto com o \textit{template Vuexy}, revelou-se eficaz para permitir a rápida criação de interfaces modernas. No entanto, em ambientes de produção, uma abordagem mais modular teria sido benéfica para agilizar a manutenção futura e permitir a integração de novas funcionalidades.

A base de dados relacional \textit{MySQL} foi suficiente para o desenvolvimento da prova de conceito, mas para uma maior flexibilidade, particularmente com a diversidade das práticas ESG entre setores e empresas, sistemas como \textit{PostgreSQL} (com suporte a campos dinâmicos) ou soluções \textit{NoSQL} poderiam ser mais adequados.

No caso das infraestruturas, a utilização do \textit{AWS S3} facilitou o desenvolvimento local e prototipagem rápida. Apesar de esta solução ter sido adequada, numa implementação eficiente, a utilização do \textit{Azure Blob Storage} teria permitido uma integração mais natural com o ecossistema tecnológico já existente na empresa, facilitando futuras configurações e manutenções.

Em relação à fundamentação teórica, a comparação das principais \textit{frameworks} ESG, GRI e SASB, permitiu uma decisão acertada na adoção do SASB, que já é utilizado pela empresa. No entanto, a plataforma poderia ter sido concebida desde o início com mais abstração para suportar várias \textit{frameworks}, melhorando assim a escalabilidade e a reutilização em vários contextos empresariais.

Uma comparação de soluções empresariais como \textit{Workiva}, \textit{IBM ESG Suite} e \textit{SAP Sustainability Control Tower} contribuiu para a definição funcional da aplicação. Uma investigação técnica mais aprofundada das suas API, fluxos de dados e práticas de integração teria enriquecido ainda mais a arquitetura, nomeadamente na forma como os objectivos são geridos.

Em suma, a aplicação desenvolvida atingiu todos os objectivos para esta fase de prova de conceito. Melhorias estratégicas e tecnológicas adicionais, alinhadas com o contexto organizacional e as exigências de escalabilidade, poderiam também ajudar na sua manutenção e evolução a longo prazo.

\subsection{Considerações Pessoais}

O planeamento e desenvolvimento do projeto, no âmbito profissional da DevScope, foram extremamente proveitosos. O ambiente positivo de troca de conhecimento, apoio e consistência contribuiu significativamente para a efetiva realização e implementação bem sucedida das funcionalidades propostas.

A nível pessoal, a experiência foi especialmente gratificante. O tema do projeto desperta um grande interesse por parte do autor, o que aumentou o envolvimento. Além disso, o desafio foi uma oportunidade de resiliência e crescimento, incluindo interações com profissionais experientes do setor.

A empresa forneceu um estímulo constante e \textit{feedback} pertinente, inclusive na redação do relatório.

Para concluir, o projeto foi uma experiência muito enriquecedora, tanto do ponto de vista técnico como humano, que abriu portas para futuras possibilidades profissionais e reforçou competências básicas para a integração no mercado de trabalho.